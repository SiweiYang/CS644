\documentclass[12pt,letterpaper]{article}
\usepackage[margin=0.75in]{geometry}
\usepackage[pdftex]{graphicx}
\usepackage[parfill]{parskip}
\usepackage{listings}
\usepackage{color}
\lstset{language=Haskell, frame=single, breaklines=true, keywordstyle=\color{blue}, identifierstyle=\color{red}, basicstyle=\small}

% Submit to Marmoset a PDF document of no more than five pages explaining the design of this phase of your compiler.
% The document should be organized to enable someone unfamiliar with your code to understand the structure of (this phase of) your compiler.
% In the document, discuss challenges that you encountered and how you tried to overcome them in your design and implementation.
% Also explain the testing that you did before submitting to Marmoset.

\begin{document}

\begin{titlepage}
\begin{center}

% Title
\Large \textbf { {\bf CS644 Project} \\ Report: A5}\\[0.5in]

       \small \emph{Submitted in partial fulfillment of\\
        the requirements for the course}
        \vspace{.2in}

% Submitted by
\normalsize Submitted by \\
\begin{table}[h]
\centering
\begin{tabular}{lr}\hline \\
Name & ID\\ \\ \hline
\\
Siwei Yang & s8yang \\ \\ \hline
\\
Justin Vanderheide & jtvander \\ \\ \hline
\\
Jian Li & j493li \\ \\ \hline
\end{tabular}
\end{table}

\vfill

% Bottom of the page
\includegraphics[width=0.3\textwidth]{../res/UWLogo}\\[1.1in]
\Large{Department of Computer Science}\\
\normalsize
\textsc{University of Waterloo}\\
Waterloo, Ontario, Canada -- N2L 3G1 \\
\vspace{0.2cm}
Winter 2015

\end{center}
\end{titlepage}

\section{General Strategy}
This section describes the meta strategy we applied in developing the code generation phase of our compiler.

In principle, we divide the code generation into two stages where at the end of each stage we can look at the results to catch any mistakes. This not only saved us debugging effort, but also helped managing the complexity of assemly code generation. As looking at the assembly code is not an effective way of debugging the compiler, we tried minimize the need to delve into assembly code by making the generation a pretty mechanical process where the correctness can be easily checked and proved.

The key to trivial assembly generation is refining our AST representation yet again. We first distill the necessary information from the attribute grammar into DFExpression, which have as much necessary information gathered as possible.

The algebraic datatype for DFStatement and DFExpression is the following.
\begin{lstlisting}
data DFExpression = FunctionCall Symbol [DFExpression]
                   | ArrayAccess Symbol DFExpression DFExpression
                   | Unary { op :: String, expr :: DFExpression }
                   | Binary { op :: String, exprL :: DFExpression, exprR :: DFExpression }
                   | Attribute { struct :: DFExpression, mem :: Symbol }
                   | InstanceOf { reftype :: Type, expr :: DFExpression }
                   | Cast {reftype :: Type, expr :: DFExpression }
                   | ID { identifier :: Either Int (Int, Symbol) }
                   | Value { valuetype :: Type, value :: String }
                   | Super { offset :: Int, super :: Maybe Symbol }
                   | Null
                   | NOOP
\end{lstlisting}

Note the change of strings representing \emph{names} being converted to our internal type \emph{Symbol}. The \emph{Symbol} is a data type uniquely represents entities of the source program: \emph{Type}, \emph{Method}, \emph{Field}. In short, all the type linking and name resolution results are condensed into the new \emph{DFExpression} now.

\begin{lstlisting}
data DFStatement = DFIf {
  condition :: DFExpression,
  ifBlock   :: [DFStatement],
  elseBlock :: [DFStatement],
  nesting   :: [Int]
} | DFWhile {
  condition  :: DFExpression,
  whileBlock :: [DFStatement],
  nesting    :: [Int]
} | DFFor {
  initializer :: DFStatement,
  condition   :: DFExpression,
  finalizer   :: DFStatement,
  forBlock    :: [DFStatement],
  nesting     :: [Int]
} | DFBlock {
  block :: [DFStatement],
  nesting :: [Int]
} | DFExpr DFExpression
  | DFLocal DFExpression
  | DFReturn (Maybe DFExpression)
\end{lstlisting}

All of the assembly generation related code is located in \emph{CodeConstruct.hs}.

\subsection{Unary and Binary Operations}
Our approach to compiling operators was to use eax to store the left hand side and result, and use ebx to store the right hand side and result.
Precedence aside, expressions are then evaluated left to right, and this would also take into account the side effect of evaluation order.
The left hand side is evaluated and pushed to the stack, the right hand side is then evaluated and the result sits in eax, then move eax to ebx, then the left hand side is popped from the stack to eax, and the operation runs, leaving the final result in eax.
For unary operations there is simply no ebx, the single operand is stored in eax, and the result replaces it.

\subsection{Variables}
There are effectively 4 different types of variables that we need to handle.
The easiest are parameter variables. These are accessed based on their offset from the base pointer.
Since arguments are pushed onto the stack in reverse order followed by the return address and saved ebp, the first parameter is in [ebp+8] following by [ebp+12] and so on.

Local variables are placed on the stack in the order they are defined. The first local variable is then located in [ebp-4], followed by [ebp-8] and so on.
For local variables within nested blocks we also push them to the stack, and whenever we exit a block we pop them off the stack.

Static variables live in the data segment. They are accessed by their unique label.

Instance variables are accessed by an offset from the variables memory address, The first variable resides at [varAddr+4], then [varAddr+8] and so on.

\subsection{Function Calls}
For function calls, the arguments are evaluated left to right and pushed to the stack.
In the case of a static function the x86 'call' instruction is then used to store the base pointer and jump to the function.
For instance methods the target function address is first retrieved from the vtable before being called.
The address of the object, i.e. this pointer, is pushed as the first hidden argument.
Our convention is to have the callee clean up the stack, so after the function call returns, the callee pops the arguments.

\subsection{Static Initialization}
For each class, we need first initialize all static fields at the beginning of program.
So we put these initializing code separately into each individual assembly output file, and call them at the entrance point in output file \emph{main.s}.
However, we allocate all space together using \emph{.data} section in file \emph{main.s}, and given each field an unique label to find it.
There is one more challenge here: some static fields would use other static fields to initialize themselves, so different initializing order could give different results.
To address this issue, we analyze the initialing dependencies of static fields in all classes, and build a dependency graph to give a topological order to initialize.

\subsection{Virtual Function Table}
Our vtable generation is handled by the function \emph{genAsmVirtualTable}.
Since we have a unique integer for each function in the source being compiled, generating the vtable is actually quite easy.
We iterate through all instance functions in the program and then look up the corresponding label for the classes implementation.
If the class does not support a given label we put the label for \_\_exception to fill the gap, since that function should never be called on objects of the class.
At runtime when a method is called we use it's unique integer as an offset from the base of the vtable pointer in the object.

Here is the genAsmVirtualTable function in it's entirety.
\begin{lstlisting}
genAsmVirtualTable :: SymbolDatabase -> ClassConstruct -> [String]
genAsmVirtualTable sd (CC _ _ sym _) = header ++ code
  where
    header = ["__vft:"]
    [(_, vtable)] = filter (\(symbol, _) -> symbol == sym) (instanceFUNCTable sd)
    labelT = map (\symbol -> if isNothing symbol then "__exception" else (funcLabel sd) ! (fromJust symbol)) vtable
    code = map (\str -> "dd " ++ str) labelT
\end{lstlisting}

\subsection{Arrays}
To simplify the implementation of arrays, we wrote an array implementation in Java which can be found in \emph{res/Array.java}.
The implementation has two instance variables: a base pointer and a length.
The length is used to throw runtime exceptions for out of bounds access, while the pointer is used for accessing elements.
Array creations are transformed into calls to either of the array constructors, and accesses are transformed into calls to the get instance method.
By creating out array implementation in Java, arrays were no extra trouble to handle versus any other data type.

\section{Challenges}
Last but not least, we would like to cover some technicalities involving the following challenges. Each further demonstrates how we model the solution, and how we engineer solution taking into account the characteristics of pure functional language.

\subsection{Unique Label Creation}
It is necessary to have many unique labels throughout the generated assembly for control flow blocks as well as simple constructs like short circuiting boolean expressions.
Since Haskell is pure it is not possible for us to simply have a global counter that we can rely on to generate these unique labels.
Instead, we keep track of the nesting of statements and expressions.
A nesting is a list of integers, where the last number is the order of the statement within the parent, and the prefix is the parents nesting.
This ensures that within each function the nestings are unique, we then use nasms local label feature to ensure that labels in different functions do not clash.
If we were to do it all again, we would use the state monad to handle this, maintaining the nesting value throughout generation was quite bothersome.

Function labels are simply a concatenation of the package name, class name, the word instance or static, followed by a globally unique integer that identifies the function.
By generating our function labels this way our assembly is still readable, and we can be sure that there are no conflicts.

%While generating new DFExpressions from Expression, we also generate some unique labels for assembly code.
%For example, the labels for If, While, For, as well as short-circuiting for logical operator $\&\&$ and $||$.
%The reason that generation unique label can be thought as a challenge is we are using purely functional programming language, so that we cannot maintain a global label generator or counter to do this.
%Therefore, we use a list of numbers, namely \emph{nesting}, to encode current unique position in AST, and append these unique labels to statements and expression which need it.

\subsection{Constructor}

\subsection{Virtual Function Table}
Dynamic dispatch on function calls is one of the most important feature provided by OOP languages. However, because of multiple inheritance(from interface for Joos), the implementation is not as straightforward as it sounds. For each \emph{Class}, the dynamic dispatch can be described by a list of pairs where each pair contains the function being called and the actual implementation to use.

So, if correctness is the only concern, then we can run a linear scan on each dynamic function call to get the correct implemention for it. But we also want our program to be reasonably performant. Therefore, we would much rather do a offset based solution which requires only one redirection upon reading the contents for the virtual function table. However, multiple inheritance implies we can not guarantee a offset for a function declaration by reordering all the implemented functions.

In class we have discussed various ways to fix the ordering issue. But we decided to go with a much simpler and performant solution. At the cost of increasing program size. Our approach can be illustrated by the following code.

\begin{lstlisting}
__vft:
dd function_String_instance_toString_46
dd function_String_instance_toString_46
dd function_String_instance_toString_46
dd __exception
dd function_Object_instance_clone_24
dd __exception
dd function_String_instance_hashCode_41
\end{lstlisting}

Notice that on the virtual function table there is a slot not being filled, but goes to exception. That is the important relaxation we made so that a consistent offset can be decided for all function declarations. So an object can get the same function calls from all its supported interfaces. And all of those calls have to be translated into different offsets. But the virtual function table have entries for all of the offsets so that the call can be resolved correctly at runtime.

\subsection{Saving inheritance info to assembly file}
Since joosc language supports \emph{instanceOf} and \emph{casting} features, we cannot determine the type of an object at compile-time.
All of inheritance info should be stored in assembly code constantly, and look it up when doing \emph{instanceOf} and \emph{casting}.
In the assembly source file \emph{main.s}, starting from label \emph{\_\_characteristics:}, there is a large bitmap.
To look up a relation is true or not, let $a$ be the class id stored in an object, and $b$ be the class id generated at the compile-time in terms of given class name.
The offset is $a \cdot total + b$, where $total$ is the number of classes, including stdlibs.
And if that bit is 1, the relation is true.
When doing object casting, we use the same procedure, but if result is 0, which means this conversion is invalid, just call \emph{\_\_exception} and throw an exception.

\section{Debugging and Testing Strategies}
For assignment 5 we wrote a test runner in bash.
The bash script automatically runs all of the marmoset tests locally, as well as tests we added, from beginning to end.
The script prints out all of the tests we are currently failing, and in order to assist in debugging it indicates if a failure occured during compilation, assembly, linking, or actual execution.
Just as in assignment 4 the script is broken up into various functions which are exported to provide us with convenient ways of invoking joosc.
For example the function "joosc\_run" takes in the path to either a single file test or a test directory and runs joosc with the appropriate standard library files and test directory files.
Unlike the previous assignment we no longer run the tests in parallel because of contention for the output directory.
While we could have set up one output directory for each thread, we wanted the execution of our tests to mimic marmoset as closely as possible.
\end{document}

